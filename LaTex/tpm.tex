\documentclass[resume]{uboot}


\begin{document}
\section{TPM}

\subsection{Savoir expliquer le principe des chiffrements symétrique, asymétrique, fonctions de hachage, la signature digitale.}

\subsubsection{symétrique}

- Une seul clé pour crypter et décrypter
- Codage par bloc ou par bloc chainé **(figure CBC)** 
- `openssl enc -aes-256-cbc -e -in t.txt -out t.enc` //encrypt
- `openssl enc -aes-256-cbc -d -in t.enc -out t.txt`//decrypt

\subsubsection{asymétrique}

- Deux clés (publique et privée) clé publique disponible par des certificats (CMD pgp)
- Encrypt public -> Decrypt private => confidentialité
- Encrypt private-> Decrypt public => signature digitale

\subsubsection{hash}

Transforme un texte, document en un nombre de N bits unique (SHA-2, SHA-3, Blake2).

`md5sum file => a6a0e8d0522e2c5de921b1c455506320` ou `openssl dgst -md5 file MD5(file)= a6a0e8d0522e2c5de921b1c455506320`  

\subsubsection{signature}

En deux parties: 1. Calcul du HASH puis encryptage avec clé privée. **(figure signature)**

\subsection{Connaitre les différentes implémentations des TPM (discrete, integrated, Hypervisor, Software)}

- discrete : Circuit dédié 
- integrated : Partie du $\mu C$ qui gère le TPM
- Hypervisor : virtuel fournis par personne fiable
- Software : virtuel pour faire des test pas sécurisé

\subsection{Connaitre l’architecture interne d’un TPM}

**(figure internal)** 

\subsection{Connaitre les différentes hiérarchies des TPM (endorsement, platform, owner, null)}

- endorsement : réservé au fabricant du TPM et fixé lors de la fabrication.
- platform : réservé au fabricant de l'hôte et peut être modifier par l'équipementier.
- owner : hiérarchie dédiée à l'utilisateur primaire du TPM peut être modifié en tout temps.
- null : réservé aux clés temporaires (RAM s'efface à chaque redémarrage)

\subsection{Savoir créer, utiliser des clés avec un TPM}

**(figure createKey)** 

\subsection{Connaitre les commandes principales d’un TPM}

- `tpm2_createprimary -C o -G rsa2048 -c o_prim` // créer un clé primaire owner
- `tpm2_getcap handles-transient` // voir clé dans la RAM
- `tpm2_getcap handles-persistent` // voir clé dans la NV-RAM
- `tpm2_evictcontrol –c o_primary.ctx` // sauver une clé en NV-RAM
- `tpm2_flushcontext` -t // effacer toute la RAM
- `tpm2_create -C o_prim -G rsa2048 -u child_pub –r child_priv` // créer clé enfant
- `tpm2_load -C o_prim -u child_pub -r child_priv -c child` // charger clé enfant


\end{document}