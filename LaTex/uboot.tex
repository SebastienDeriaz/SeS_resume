\documentclass[resume]{subfiles}


\begin{document}
\section{U-boot}
\subsection{Compilation}
On configure avec \verb!make ubiit-menuconfig! puis on effectue la compilation avec une des deux manières :
\begin{enumerate}
\item \verb!make uboot-rebuild!
\item supprimer les fichiers puis \verb!make!
\end{enumerate}
\subsection{Démarrage}
Si on appuie sur une touche, on entre en mode u-boot. La commande \verb!booti! permet de lancer l'image linux (\verb!boot! tout court va aussi lancer l'image Linux).\\
Avec les commandes présentes dans \verb!boot.cmd!, on indique l'emplacement dans la ram de \verb!Image! et \verb!nanopi-neo-plus.dtb!


\end{document}