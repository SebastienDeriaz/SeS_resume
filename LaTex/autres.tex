\documentclass[resume]{subfiles}



\begin{document}
\section{Autres}
\subsection{Commandes}
\begin{table}[H]
\centering
\begin{tabular}{ll}
Commande & Description\\\hline
\verb!netcat! (\verb!nc!) & Couteau suisse du TCP/IP. Permet de scanner des ports\\
\verb!nmap! & Analyse des ports ouverts\\
\verb!ssh! & Connexion à un système par interpréteur de commande\\
\verb!dd! & copie byte à byte entre des streams. (\verb!sudo dd if=/dev/zero/ of=/dev/null bs=512 count=100 seek=16!)\\
\verb!parted! & création / modification de partiations (\verb!sudo parted /dev/sdb mklabel msdos!\\
\verb!mkfs.ext4! & commandes ext4 pour créer / modifier une partition
\end{tabular}
\end{table}

\subsection{Définitions}
\begin{table}[H]
\begin{tabular}{ll}
Nom & Description\\\hline
Honeypot & "Pot de miel" ou leurre pour faire croire qu'un système non-sécurisé est présent (à tord)\\
Toolchain & Codes sources et outils nécessaires pour générer une image éxécutable (sur un système embarqué)\\
Kernel & Coeur Linux (avec le format u-boot)\\
Rootfs & Root Filesystem (avec tous les dossiers et outils utilisés par Linux)\\
Usrfs & User Filesystem (applications spécifiques à l'utilisation du système embarqué)\\
Buildroot & Ensemble de makefiles et patchs qui simplifient et automatisent la création d'un Linux pour système embarqué\\
uClibc & Librairie c de base similaire à glibc mais plus compacte (pour systèmes MMU-less)\\
Busybox & Binaire unique qui contient toutes les commandes de base (ls, cat, mv)
\end{tabular}
\end{table}
\end{document}