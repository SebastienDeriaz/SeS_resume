\documentclass[resume]{subfiles}


\begin{document}
\section{Hardening}

\subsection{Intégrité package, programme}
Aller voir la section \ref{sec_sign}.
\begin{lstlisting}[style=console]
gpg --verify "package"
gpg --keyserver keyserver.ubuntu.com --search-keys "KEY"
\end{lstlisting}

\subsection{Configurer un package, programme}
\begin{lstlisting}[style=console]
tar xvzf package1.tar.gz
cd package1
less INSTALL or less README #Analyze the different options
./configure --help
\end{lstlisting}

\subsection{Cross-compiler un programme}
Ajout de host et prefix
\begin{lstlisting}[style=console]
./configure --host=aarch64-none-linux-gnu --prefix=/home/dir
make 
make install
\end{lstlisting}

\subsection{Contrôler les services, les ports ouverts}
\begin{itemize}
\item ps -ale : montre tous les process
\item ps -aux : montre les droits des process
\item lsof    : montre les ports ouverts
\item nmap    : montre les ports ouvert sur à une IP

\end{itemize}


\subsection{De contrôler les file systems}
\textcolor{red}{A FAIRE}

\subsection{Permissions des fichiers, dossiers}
\begin{lstlisting}[style=console]
ls -al => -rwxrwxrwx usr grp ..... t.txt
chmod 755 t.txt => -rwxr-xr-x usr grp .....t.txt
\end{lstlisting}

\subsection{Sécuriser le réseau}
\begin{itemize}
\item Désactiver l'IPv6
\item Désactiver le routage source IP
\item Désactiver le port forwarding
\item Bloquer la redirection des msg ICMP
\item Activer la vérification de routage source
\item Log paquet erroné et ignore bogus ICMP
\item Désactiver ICMP echo et temps
\item Activer syn cookies (pour TCP)
\end{itemize}

\subsection{Contrôler-sécuriser user}
Modifier le umask à 0027 réduit les droits
$Droit = \overline{umask}\space \& 0777$

\subsection{Limiter le login root}
\begin{lstlisting}[style=console]
chmod 700 /root #limite l'accès au dossier root
sudo #pour avoir les droits root
\end{lstlisting}
Ne pas mettre le . dans la path

\subsection{Sécuriser le noyau}
Aller voir dans la section \ref{sec_protections}

\subsection{Sécuriser une application}
Activer l'option de compilation $-fstack-protector-all$ et $noexecstack$
\begin{lstlisting}[style=console]
gcc -Wall -Wextra -z noexecstack -pie -fPIE -fstack-protector-all -Wl,-z,relro,-z,now -O -D_FORTIFY_SOURCE=2 -ftrapv -o test test.c
\end{lstlisting}

\end{document}